%==============================================================================
% Sjabloon onderzoeksvoorstel bachproef
%==============================================================================
% Gebaseerd op document class `hogent-article'
% zie <https://github.com/HoGentTIN/latex-hogent-article>

% Voor een voorstel in het Engels: voeg de documentclass-optie [english] toe.
% Let op: kan enkel na toestemming van de bachelorproefcoördinator!
\documentclass{hogent-article}

% Invoegen bibliografiebestand
\addbibresource{voorstel.bib}

% Informatie over de opleiding, het vak en soort opdracht
\studyprogramme{Professionele bachelor toegepaste informatica}
\course{Bachelorproef}
\assignmenttype{Onderzoeksvoorstel}
% Voor een voorstel in het Engels, haal de volgende 3 regels uit commentaar
% \studyprogramme{Bachelor of applied information technology}
% \course{Bachelor thesis}
% \assignmenttype{Research proposal}

\academicyear{2025-2026} % TODO: pas het academiejaar aan

% TODO: Werktitel
\title{Ontwerp en realisatie van een performante provisioner voor een virtuele Windows- en Linux omgevingen}

% TODO: Studentnaam en emailadres invullen
\author{Jens Gosseye}
\email{jens.gosseye@student.hogent.be}

% TODO: Medestudent
% Gaat het om een bachelorproef in samenwerking met een student in een andere
% opleiding? Geef dan de naam en emailadres hier
% \author{Yasmine Alaoui (naam opleiding)}
% \email{yasmine.alaoui@student.hogent.be}

% TODO: Geef de co-promotor op
\supervisor[Co-promotor]{/ (/, \href{mailto:/}{/})}

% Binnen welke specialisatierichting uit 3TI situeert dit onderzoek zich?
% Kies uit deze lijst:
%
% - Mobile \& Enterprise development
% - AI \& Data Engineering
% - Functional \& Business Analysis
% - System \& Network Administrator
% - Mainframe Expert
% - Als het onderzoek niet past binnen een van deze domeinen specifieer je deze
%   zelf
%
\specialisation{System \& Network Administrator}
\keywords{Provisioning, Linux, Windows}

\begin{document}

\begin{abstract}
    In deze bachelorproef wordt onderzocht hoe dat je een provisioner realiseert die dat gebruikt kan worden voor virtuele Windows- en Linuxomgevingen. Dit zou een alternatief zijn voor de huidige Vagrant provisioner die dat  gebruikt wordt in de opleidingen van toegepaste informatica aan HOGENT. De centrale onderzoeksvraag is: ``Hoe ontwerp en realiseer je een preformante provisioner voor een virtuele Windows- en Linux omgevingen, die dat beter inspeelt op de behoeften van onderwijs en bedrijven dan de huidige Vagrant-gebaseerde aanpak?''. Hierdoor wordt het probleem van de soms trage en onbetrouwbare Vagrant provisioner aangepakt.
\end{abstract}

\tableofcontents

% De hoofdtekst van het voorstel zit in een apart bestand, zodat het makkelijk
% kan opgenomen worden in de bijlagen van de bachelorproef zelf.
%---------- Inleiding ---------------------------------------------------------

% TODO: Is dit voorstel gebaseerd op een paper van Research Methods die je
% vorig jaar hebt ingediend? Heb je daarbij eventueel samengewerkt met een
% andere student?
% Zo ja, haal dan de tekst hieronder uit commentaar en pas aan.

%\paragraph{Opmerking}

% Dit voorstel is gebaseerd op het onderzoeksvoorstel dat werd geschreven in het
% kader van het vak Research Methods dat ik (vorig/dit) academiejaar heb
% uitgewerkt (met medesturent VOORNAAM NAAM als mede-auteur).
% 

\section{Inleiding}%
\label{sec:inleiding}

Binnen de richting IT infrastructure engineer wordt in HOGENT gebruikgemaakt van verschillende provisioningtools om virtuele Windows- en Linuxomgevingen op te zetten. Een tool dat hierbij veel gebruikt wordt is Vagrant, studenten gebruiken dit om een ontwikkel- en labo-omgeving te kunnen automatiseren op basis van een declaratieve configuratiefile. In de praktijk komen er verschillende problemen tevoorschijn: de virtuele machines starten soms traag op, het provisioningproces hangt vast of kan niet gestart worden door time-outs, virtuele machines kunnen onverwachte kernel panics vertonen en als het provisioningproces onderbroken wordt tijdens uitvoer blijft het soms verder draaien op de achtergrond waardoor dat de gebruiker dit proces manueel moet stoppen om verder te kunnen werken. Dit zorgt voor vele frustraties en tijdverlies zowel binnen een onderwijscontext of bedrijfscontext. 

De centrale onderzoeksvraag van deze bachelorproef is: ``Hoe ontwerp en realiseer je een preformante provisioner voor een virtuele Windows- en Linux omgevingen, die dat beter inspeelt op de behoeften van onderwijs en bedrijven dan de huidige Vagrant-gebaseerde aanpak?'' Om deze vraag te beantwoorden doen we aan de hand van deze deelvragen:
\begin{itemize}
  \item Welke programmeertaal is het meest geschikt om een provisioner te schrijven? 
  \item Welke functionaliteiten zijn nodig voor een provisioner?
  \item Op welke besturingssystemen zal de provisioner gebruikt worden?
  \item Met welke virtualisatie software zouden we het beste compatibiliteit bieden?
\end{itemize}

Dit onderzoek streeft ervoor om een provisioner te maken die dat gebruikt kan worden door IT-professionals en studenten binnen een onderwijs- en labo-omgevingen. 


%---------- Stand van zaken ---------------------------------------------------

\section{Literatuurstudie}%
\label{sec:literatuurstudie}


Virtuele machines vormen de basis van veel Informaticatoepassingen. De voornaamste reden hiervoor is omdat ze het mogelijk maken om meerdere virtuele omgevingen te laten draaien op dezelfde fysieke hardware en zo de flexibiliteit, schaalbaarheid en resource-efficiëntie verhogen.\autocite{Ahmed2020} In verschillende surveys wordt \emph{provisioning} beschreven als het geautomatiseerd aanmaken, configureren en beheren van een virtuele machine.\autocite{Kirana2018,Abdulhamid2014} Zoals deze studies aangeven is het opzetten van een testomgeving met efficiënte provisioning-strategieën cruciaal bij het uitwerken van een project.

\subsection {Infrastructure as Code}
Het automatiseringsproces wordt in systeembeheer genoemd \emph{Infrastructure as Code (IaC)}. Verschillende studies en rapporten binnen het IaC een maken onderscheid tussen verschillende soorten tools, zoals emph{provisioning tools}, \emph{configuration management} en \emph{orchestration} \autocite{Hill2015,Yevgeniy2025}. Provisioningtools  richten zich op het creëren van de infrastructuur, configuratiemanagement tools richten zich op de installatie en configuratie van software op de server \autocite{Ijaz2024}. 

Het landschap van deze verschillende tools is divers. Voor de lokale ontwikkelingsomgevingen wordt gebruikt gemaakt van tools zoals Vagrant. Verschillende tools bestaan ook die dat niet gemaakt zijn als eerste instantie voor lokale omgevingen zoals Ansible en Terraform, deze worden vaak geassocieerd met cloud omgevingen of grote configuraties \autocite{Ismailzai2017}. Recent vergelijkingen tussen deze verschillende soorten tools toont aan dat de grenzen soms vervagen, de keuze voor een specifiek tool hangt af van de use-cases \autocite{Slattery2025}.


\subsection{Educatieve context}
Alhoewel dat er veel verschillende commerciële oplossingen bestaan voor enterprise-\\omgevingen \autocite{Portworx2025}, is er binnen een onderwijscontext andere verschillende eisen voor virtualisatie. Onderzoek naar het gebruik van virtuele machines (VM's) in het onderwijs toont dat een balans tussen performantie, gebruiksvriendelijkheid voor studenten en de beperkingen van consumenten hardware zoals laptops een complexe uitdaging is \autocite{Robison2012}.

Ook speelt tijd een grote rol. De \emph{provisioning time}, de tijd die tussen het aanvragen van de vm en beschikbaarheid ervan is kritieke performance-indicator (KPI) die dat productiviteit beïnvloed \autocite{KpiDepot2024}. Lange provisioning time kan tijdens het maken van een labo of examen voor minder werktijd zorgen. Gelukkig zijn er wel verschillende algoritmen die dat dit kunnen versnellen \autocite{jcai2020p208,Lo2014}, blijft de vraag bestaan hoe je een lichtgewichte en snelle provisioner kan gerealiseerd worden die dat specifiek gemaakt is voor Windows- en Linux-omgevingen voor een educatieve omgeving als alternatief voor de huidige Vagrant provisioner.





% Voor literatuurverwijzingen zijn er twee belangrijke commando's:
% \autocite{KEY} => (Auteur, jaartal) Gebruik dit als de naam van de auteur
%   geen onderdeel is van de zin.
% \textcite{KEY} => Auteur (jaartal)  Gebruik dit als de auteursnaam wel een
%   functie heeft in de zin (bv. ``Uit onderzoek door Doll & Hill (1954) bleek
%   ...'')

%---------- Methodologie ------------------------------------------------------
\section{Methodologie}%
\label{sec:methodologie}




\subsection{Fase 1: Vereisten verzamelen}
In de eerste fase worden de functionele en niet-functionele requirements in kaart gebracht, dit zal gedaan worden op basis van de huidige problemen met Vagrant en de vereisten van onderwijs en testomgevingen. Hierbij zal bepaald worden welke besturingssystemen we zullen ondersteunen voor het opzetten van de virtuele omgeving, welke virtualisatie software relevant is en welke programmeertaal dat het meest optimaal zou zijn voor het schrijven van een provisioner.

\subsection{Fase 2: Programmeertaal bestuderen}
In de tweede fase wordt de focus gelegd op het aanleren van de gekozen programmeertaal. Hierbij zullen de basisconcepten van de taal geleerd worden, zoals functies,methodes,datatypes en algemene syntax. Aan de hand van verschillende kleine projecten worden deze concepten dan toegepast en aangeleerd.

\subsection{Fase 3: Programmeren van de software}
In de derde fase wordt de provisioner ontwikkeld op basis van de verzamelde vereisten. In het begin zullen de kernfunctionaliteiten worden geïmplementeerd, zoals het opstarten van de virtuele machines, het lezen van de configuratiefiles en het uitvoeren van de provisioning scripts. Daarna zullen verschillende extra functionaliteiten toegevoegd worden, ook zal er foutafhandeling en logging komen zodat de tool stabiel kan functioneren.

\subsection{Fase 4: Testen}
In de vierde fase wordt de provisioning software getest aan de hand van verschillende scenario's, hierbij zal de software getest worden op normaal- en grensgebruik. Tijdens deze testen zal op een objectieve manier zoals opstarttijd, totale provisioningtijd, aantal fouten en de noodzaak tot manuele interventie gemeten worden, zodat duidelijk wordt in welke mate de software robuust en betrouwbaar functioneert.


\subsection{Fase 5: Rapportering}
In de vijfde fase worden resultaten van de testen verwerkt in de bachelorproef en andere bijhorende documentatie. Ook zal er een gebruikshandleiding geschreven worden die dat gebruikers kunnen gebruiken.

\begin{figure*}
    \centering
    \includegraphics[width=\textwidth]{img/Gantt Diagram.png}
    \caption{\label{fig:gantt}Gantt diagram met de verschillende fasen en milestones van het onderzoek.}
\end{figure*}
\newpage
\subsection{Planning}
De totale duur van deze methodologie is gepland over een periode van 12 weken.
\begin{itemize}
    \item Week 1: Vereisten verzamelen
    \item Week 2-3: Leren van programmeer taal
    \item Week 4-9: Programmeren
    \item Week 10: Testen
    \item Week 11-12: opstellen rapportering
\end{itemize}


%---------- Verwachte resultaten ----------------------------------------------
\section{Verwacht resultaat, conclusie}%
\label{sec:verwachte_resultaten}
In dit onderzoek wordt er verwacht dat de ontwikkelde provisioner virtuele omgevingen van Windows en Linux kan opbouwen op een snellere en betrouwbaardere manier dan de huidige Vagrant provisioner. Dat de verkorte provisioningtijd,minder mislukte runs en verlaagde manuele interventie er voor zorgt dat de virtuele omgeving betrouwbaarder opstart.

Voor de doelgroep, opleidingen die dat gebruik maken van provisioning, betekent dit dat lessen, oefeningen en projecten minder tijdverlies hebben door technische problemen. De bachelorproef zal niet alleen maar een proef-of-\\concept provisioner maken maar ook documentatie en een handleiding voor mogelijkheid van uitbreiding





\printbibliography[heading=bibintoc]

\end{document}